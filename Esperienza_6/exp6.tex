% Un articolo scritto con LaTeX
\documentclass[a4paper,11pt]{article}
\usepackage{graphicx}
\usepackage[T1]{fontenc} % codifica dei font in uscita
\usepackage[utf8]{inputenc} % lettere accentate da tastiera
\usepackage[italian]{babel} % lingua principale del documento
\usepackage{url}
\usepackage [a4paper, top=2.5cm, bottom=2.5cm, left=1.5cm, right=1.5cm, bindingoffset=8mm] {geometry}

% inizio documento

\usepackage{graphicx}
\begin{document}
\begin{center}
\textbf{\huge Esperienza VI} \\ \vspace{10pt}
\large Alberini Giacomo \\ Bassini Luigi \\ Michele Pedrotti\\ Trevisson Nicola 
\end{center}
\section{Scopo dell'esperienza}
Lo scopo di questa esperienza è quello di misurare la conduttanza di vari tubi con diametri e lunghezze diversi in regimi di flusso differenti.
\section{Misura della Conduttanza}

Per calcolare la conduttanza si è reso necessario l'uso di una pompa da vuoto, due pirani (con relativa strumentazione di misura ), una vlavola a spillo (già tarata in precedenza) e vari tubi con lunghezze e diametri differenti.

	
\begin{center} 
\begin{figure}[htpd]
\hspace{120pt}
\includegraphics[scale=0.5]{schema_finale.png}
\end{figure}
\end{center}

Nello schema soprastante, per semplicità, i vari tubi sono indicati dalla linea retta verticale tra i due T che collegano i pirani.
Per ottenre il valore di conduttanza a differenti regimi di flusso, opportunamente creati tramite la manipolazione della valvola a spillo, si sono utilizzati i due pirani. Essi ci hanno permesso di rilevare i valori di pressione in testa e in coda al tubo sotto analisi. L'andamento della pressione è stato monitorato tramite un software che ha reso osservabile il momento in cui la pressione si stabilizzava dopo i cambiamenti di flusso passanti per la spillo. Una volta stabilizzata la pressione si è proceduto all'annotazione del valore di pressione corrispondente al numero di tacche di apertura della valvola a spillo (dal vuoto nel tubo per poi crescereda 4 a 8 giri e ritorno).
Una volta ottenuti tutti i dati relativi ad ogni tubo a nostra disposizione si passa al calcolo della conduttanza. Per fare questo è necessario utilizzare i dati di flusso delle precedenti esperienze relativi alla valvola a spillo. Noto il flusso Q della valvola a spillo e nota la differenza di pressione fra testa e coda $\delta p$ si può ricavare la conduttanza C tramite l'ugualianza $ C = \frac{Q}{\Delta P} $.\\
 

\begin{center} 
\begin{tabular}{|c|c|c|c|}
\hline Apertura valvola a spillo & Conduttanza & Regime di flusso & Conduttanza teorica \\ 
\hline 4 & 0.0015 $\cdot10^{-4}\pm 0.0006\cdot10^{-4}$ & transitorio & 0.0046$\cdot10^{-3}\pm 0.0003\cdot10^{-3}$ \\ 
\hline 5 & 0.010$\cdot10^{-4}\pm 0.007\cdot10^{-4}$ & transitorio & 0.0055$\cdot10^{-3}\pm 0.0006\cdot10^{-3}$ \\ 
\hline 6 & 0.06$\cdot10^{-4}\pm 0.01\cdot10^{-4}$ & transitorio & 0.011$\cdot10^{-3}\pm 0.001\cdot10^{-3}$ \\
\hline 7 & 0.24$\cdot10^{-4}\pm 0.05\cdot10^{-4}$ & laminare 10.4$\pm0.1$ & 0.045$\cdot10^{-3}\pm 0.003\cdot10^{-3}$ \\
\hline 8 & 0.68$\cdot10^{-4}\pm 0.09\cdot10^{-4}$ & laminare 63$\pm1$ & 0.10$\cdot10^{-3}\pm 0.01\cdot10^{-3}$ \\ 
\hline 
\end{tabular}\\
\vspace{5pt}
Tubo di lunghezza $8m$ e diametro $4mm$ .
\\
\vspace{15pt}
\begin{tabular}{|c|c|c|c|}
\hline Apertura valvola a spillo & Conduttanza & Regime di flusso & Conduttanza teorica \\ 
\hline 4 & 0.0011$\cdot10^{-3}\pm 0.0001\cdot10^{-3}$ & transitorio & 0.0070$\cdot10^{-3}\pm 0.0008\cdot10^{-3}$ \\ 
\hline 5 & 0.0059$\cdot10^{-3}\pm 0.0006\cdot10^{-3}$ & transitorio & 0.0086$\cdot10^{-3}\pm 0.0007\cdot10^{-3}$ \\ 
\hline 6 & 0.023$\cdot10^{-3}\pm 0.0002\cdot10^{-3}$ & transitorio & 0.032$\cdot10^{-3}\pm 0.0004\cdot10^{-3}$ \\
\hline 7 & 0.095$\cdot10^{-3}\pm 0.008\cdot10^{-3}$ & laminare 10.4$\pm0.1$ & 0.13$\cdot10^{-3}\pm 0.01\cdot10^{-3}$ \\
\hline 8 & 0.24$\cdot10^{-3}\pm 0.02\cdot10^{-3}$ & laminare 63$\pm1$ & 0.40$\cdot10^{-3}\pm 0.05\cdot10^{-3}$ \\ 
\hline 
\end{tabular}\\
\vspace{5pt}
Tubo di lunghezza $80cm$ e diametro $4mm$.
\\
\vspace{15pt}
\begin{tabular}{|c|c|c|c|}
\hline Apertura valvola a spillo & Conduttanza & Regime di flusso & Conduttanza teorica \\ 
\hline 4 & 0.0070$\cdot10^{-4}\pm 0.0006\cdot10^{-4}$ & transitorio & 0.0014$\cdot10^{-3}\pm 0.0001\cdot10^{-3}$ \\ 
\hline 5 & 0.027$\cdot10^{-4}\pm 0.003\cdot10^{-4}$ & transitorio & 0.0026$\cdot10^{-3}\pm 0.0002\cdot10^{-3}$ \\ 
\hline 6 & 0.095$\cdot10^{-4}\pm 0.009\cdot10^{-4}$ & laminare 1.1$\pm0.1$ & 0.011$\cdot10^{-3}\pm 0.001\cdot10^{-3}$\\
\hline 7 & 0.35$\cdot10^{-4}\pm 0.03\cdot10^{-4}$ & lamianre 16$\pm1$ & 0.049$\cdot10^{-3}\pm 0.005\cdot10^{-3}$ \\
\hline 8 & 0.94$\cdot10^{-4}\pm 0.09\cdot10^{-4}$ & laminare 102$\pm9$ & 0.12$\cdot10^{-3}\pm 0.01\cdot10^{-3}$ \\ 
\hline 
\end{tabular}\\
\vspace{5pt}
Tubo di lunghezza $80cm$ e diametro $2.5mm$.
\\
\vspace{15pt}
\begin{tabular}{|c|c|c|c|}
\hline Apertura valvola a spillo & Conduttanza & Regime di flusso & Conduttanza teorica \\ 
\hline 4 & 0.0007$\cdot10^{-4}\pm 0.0001\cdot10^{-4}$ & transitorio & 0.012$\cdot10^{-4}\pm 0.001\cdot10^{-4}$ \\ 
\hline 5 & 0.0046$\cdot10^{-4}\pm 0.0004\cdot10^{-4}$ & transitorio & 0.014$\cdot10^{-4}\pm 0.001\cdot10^{-4}$ \\ 
\hline 6 & 0.024$\cdot10^{-4}\pm 0.002\cdot10^{-4}$ & laminare 1.1$\pm0.1$ & 0.042$\cdot10^{-4}\pm 0.004\cdot10^{-4}$ \\
\hline 7 & 0.10$\cdot10^{-4}\pm 0.01\cdot10^{-4}$ & laminare 16$\pm1$ & 0.15$\cdot10^{-4}\pm 0.01\cdot10^{-4}$ \\
\hline 8 & 0.10$\cdot10^{-4}\pm 0.01\cdot10^{-4}$ & laminare 102$\pm9$ & 0.96$\cdot10^{-4}\pm 0.09\cdot10^{-4}$ \\ 
\hline 
\end{tabular}\\
\vspace{5pt}
Tubo di lunghezza $8m$ e diametro $2.5mm$.
\\
\vspace{15pt}
\begin{tabular}{|c|c|c|c|}
\hline Apertura valvola a spillo & Conduttanza & Regime di flusso & Conduttanza teorica \\ 
\hline 4 & 1.2$\cdot10^{6}\pm 0.1\cdot10^{6} $ & transitorio & 0.035 $\cdot10^{-3}\pm 0.003\cdot10^{-3}$ \\ 
\hline 5 & 0.20$\cdot10^{6}\pm 0.02\cdot10^{6}$ & transitorio & 0.042 $\cdot10^{-3}\pm 0.004\cdot10^{-3}$\\ 
\hline 6 & 0.026$\cdot10^{6}\pm 0.002\cdot10^{6}$ & transitorio & 0.091$\cdot10^{-3}\pm 0.009\cdot10^{-3}$ \\
\hline 7 & 0.0043$\cdot10^{6}\pm 0.0004\cdot10^{6}$ & laminare 5.2$\pm0.5$ & 0.32$\cdot10^{-3}\pm 0.03\cdot10^{-3}$ \\
\hline 8 & 0.0011$\cdot10^{6}\pm 0.0001\cdot10^{6}$ & laminare 31$\pm3$ & 0.92$\cdot10^{-3}\pm 0.09\cdot10^{-3}$ \\ 
\hline 
\end{tabular}\\
\vspace{5pt}
Composizione di due tubi di acciaio in serie con tre vacuometri in tre punti dell'impianto, uno in testa, uno in coda e uno in posizione intermedia fra i due tubi.
\vspace{10pt}
\end{center}
Nei conti si è tenuto in considerazione per la viscosità dinamica e per la fomula $ M/RT $ la temperatura ambientale di 20$^\circ C$.

Si può notare inoltre che i valori di conduttanza teorica sono molto maggiori dei valori calcolati. Questo è dovuto al fatto che il tubo utilizzato non è un tubo ideale ma al suo interno presenta delle imperfezioni.Inoltre i valori sono legati anche alla posizione del pirani rispetto al flusso di aria entrante nell'impianto poichè la velocità di flusso passante varia nel raccordo a t per via della presenza di un angolo retto.

Nella tabella relativa ai tubi in metallo collegati in serie abbiamo inserito l'impedenza al posto della conduttanza poichè questa è stata calcolata per i singoli tubi e quindi era necessario utilizzare la seguente formula $z=\frac{1}{c_1}+\frac{1}{c_2} $ per sommarne il risultato.



\end{document}